\chapter{Introduction}
% 1. single machine replication
% 2. state machine replication
% 2.1 weak determinism, total order of synchronization primitives.
% 3. contribution
% 3.1 deterministic execution and schedule replication inside kernel
% 3.2 fully transparent, minimal modification to application
% 3.3 comparision that sched rep is better

% cite a bunch of intra-machine solutions
% Hypervisor-based Fault-tolerance
% Operating System Support for Redundant Multithreading
% Tardigrade: Leveraging Lightweight Virtual Machines to Easily and Efficiently Construct Fault-Tolerant Services


%With increasing number of CPU cores and memory size, multi-threaded and multi-process applications are widely adopt to extract the full potential of such high performance multi-core machines. 

% make more clear statement of each paragraph

% more cpu cores can make it easier to have failures
Nowadays semiconductor industry has pushed the CPU core count to a historically high level. Having a computer system with high CPU core count and large memory capacity is cheaper than ever before. However with the technology advances, we still cannot ignore the fact that computer systems suffer from failures time to time. Given a fix probability for the failure of one CPU core, adding CPU cores leads to a higher probability of full system failure, because in current SMP operating systems the failure of a single core can bring down the entire system. A faulty device, a CPU failure or a memory damage usually lead to the entire operating system to crash. To be able to recovery from such severe failures, having backup machines is always a good idea. When the primary machine fails, the backup machines are set to be able to take over the previous work and carry on the services.

With the idea of replication, there are a good amount of works that allow users to have multiple machines to act as replicas ~\cite{guo2014rex}~\cite{kapritsos2012all}~\cite{cui2015p}~\cite{zagorodnov2009practical}~\cite{singh2009zeno}~\cite{mao2008mencius}. However, there are kinds of faults that would fail part of the OS or just the application, without breaking the entire machine. In this case full machine replication is not needed, as a result, people have tried to discover the replication inside the single machine, which we classify it as intra-machine replication. In ~\cite{zhang2012runtime} and ~\cite{lee2010respec}, the authors proposed replication systems that can have a redundant execution instance along with the original one. But doing the replication inside the same OS still cannot mitigate a transient fault that could fail the entire OS. To have a full stack replication that can minimize the impact of an OS failure, several works have investigated the approaches of doing replication via virtualization~\cite{bressoud1996hypervisor}~\cite{lorch2015tardigrade}~\cite{dunlap2002revirt}, in order to isolate the replicas in a higher degree. However, such solutions can still suffer from the faults that could crash the hypervisor. Moreover, the hypervisor based techniques usually require the replication of the entire VM state, which brings more overhead.

To achieve resilient fault-tolerance replication inside a single machine, we have to provide stronger isolation for all the replicas. Multi-kernel~\cite{baumann2009multikernel}~\cite{barbalace2014popcorn} provides the idea of running multiple OS kernels on a multi-core system without the support of a hypervisor. On a multi-kernel OS, each kernel has its own dedicated CPU cores and a part of the physical memory, the kernels can communicate with each other via messages. While multi-kernel OSes are designed to explore new means of extracting multi-core performance for concurrent applications, we found this kind of OSes can provide strong isolation that fits perfectly for intra-machine fault-tolerance replication. The Popcorn Linux~\cite{barbalace2014popcorn} project is a research effort aimed at building scalable systems software for future generations of heterogeneous and homogeneous multi/many-core architectures. As a multi-kernel OS, Popcorn Linux is able to host multiple Linux instances simultaneously on a multi-core machine. Building a replication system that utilizes multiple Linux instances on Popcorn Linux is the perfect solution for doing intra-machine replication. In this intra-machine replication model, we can treat each kernel instances as a machine node and try to adopt inter-machine replication techniques onto Popcorn Linux. Moreover, with this level of isolation, the transient fault on one kernel instance can hardly get propagated to others.

State machine replication (SMR) has been widely used for inter-machine fault-tolerance replication. In SMR, it models the services to be replicated with a set of inputs, a set of outputs and a set of states. The replication system ensures that for a given input set, from the same initial state, the replicas can produces the same state transition which in turn leads to the same output. Such a system is able to be resilient to failures in one or more replicas (depends on how many replicas are there in the system). To provide such property, determinism is required for the state machine, otherwise the state of state machines will diverge in different states even with the same input set. However, most of current SMR systems need to model the state machine into the applications explicitly. For existing applications, especially concurrent applications, it is a changellenge to transparently model the states of application for replication.

With the explorations above, we see the following challenges for doing intra-machine replication:
\begin{itemize}
\item Is it possible to adopt inter-machine replication techniques to intra-machine replication? Will we have the same performance charactristics?
\item How to transparently model existing concurrent applications for SMR replication?
\item While having less fault coverage than inter-machine replication, can intra-machine replication have less overhead?
\end{itemize}

In this thesis, on top of Popcorn Linux, we have built an intra-machine SMR system to replicate concurrent applications. In our system, we have a kernel instance as the primary replica and another kernel instance as the secondary replica. As an SMR system, we model the state of a concurrent application by its system call output and thread/process inter-leavings. We implemented two different replication modes, that are originated from previous inter-machine replication systems, to synchronize the system call output and thread/process inter-leavings. By doing so, we make sure that the replicated applications can have consistent outputs on both primary and secondary. Moreover, our system is transparent to applications, existing software can be deployed on our system with minimal modification.

% ~\cite{zhang2012runtime}
% A section for server applications
% Practical and Low-Overhead Masking of Failures of TCP-Based Servers

\section{Contributions}

Coressponds to the challenges we raised previously, this thesis presents the following contributions:

\begin{itemize}
% What are the ways to do replication? explore the performance charactristics of different replication modes we implemented blah blah...
\item To synchronize thread inter-leavings of replicated concurrent applications, based on Popcorn Linux, we implemented two different replication modes in the kernel. Both of them are originated from previous inter-machine replication solutions. Two replication modes have achieved the same goal in two different directions: Deterministic Execution uses a deterministic algorithm to decide the order of execution on both primary and secondary; while Schedule Replication enforces the secondary replica to follow the non-deterministic execution order that happened previously on the primary kernel.

% Can we support most applications transparently?
\item In order to provide transparent replication for existing applications, we provide a common programming interface for both replication modes. By wrapping a code section with our \detstart\ and \detend\ system calls, the execution order of wrapped sections can be the synchronized on both primary and secondary kernel. Based on this common interface, we also implemented a set of runtime supports that allows the user to run applications on our system with minimal code modification.

% what's the overhead of doing single machine replication?
\item To explore the pros and cons for both replication modes, and the cost of our intra-machine replication, we evaluated different types of concurrent applications on our system. For computational application we had maximum 63.39\% slowdown for Deterministic Execution and maximum 36.3\% slowdown for Schedule Replication. For two web servers we had maximum 25.22\% slowdown for Deterministic Execution and maximum 1.96\% slowdown for Schedule Replication. With the evaulation we observed that Schedule Replication is the better replication mode for intra-machine replication.

% one more bullet for the decision of replication mode
\end{itemize}

\section{Scope}
This thesis mainly discusses about how to ensure the same thread/process inter-leavings across the replicas, and some key system calls to ensure the consistent state of server applications. We do not address the non-determinism from signal, file system, random number generator and memory address space. For the replication of server applications, a member of SSRG has already implemented the replication of the TCP stack and connection recovery from kernel failures, which was able to support replicating simple single-threaded server applications prior to this work. On top of that, the work described in this thesis provides the ability of replicating real world multi-threaded/multi-process server applications.

\section{Thesis Organization}
This thesis is organized as follows:

Chapter 2 presents the background of Popcorn Linux, which is the multi-kernel system we are using for building our intra-machine replication.

Chapter 3 presents our first replication mode, Deterministic Execution. 

Chapter 4 presents our second replication mode, Schedule Replication.

Chapter 5 presents the additional runtime support that we implemented to eliminate some residual non-deterministic issues, and a runtime library to simplify the application deployment on our system.

Chapter 6 shows the performance evaluation of our system on multiple concurrent real world applications.

Chapter 7 concludes the work and discusses some future works.