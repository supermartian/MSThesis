\chapter{Popcorn Linux Background}
Our replication prototype is built on top of Popcorn Linux ~\cite{barbalace2014popcorn}. It is a multi-kernel OS which allows a multi-core system to boot multiple Linux kernels.
% talk about multi-kernel boot here
% Put a plot of popcorn architecture here
\section{Hardware Partitioning}
In Popcorn Linux, hardware resources are partitioned into arbitrary divisions, each booted kernel instance can have the full control of its own partition.

\begin{itemize}
\item{CPU Partitioning:} Popcorn Linux is able to map an arbitrary number of CPU cores to each kernel instance. In order to get the maximum performance for concurrent applications we prefer to evenly assign CPU cores to each kernel.

\item{Memory Partitioning:} By setting the starting address and memory range during the boot time of a kernel, Popcorn Linux can also partition the memory resources for all the booted kernel.
\end{itemize}

The hardware partitioning provides a very strong isolation for all the kernels and the applications running on them, which is ideal for our intra-machine fault tolerance model. When a critical hardware error happens on one kernel's hardware partition, this isolation guarantees that the error won't get propagated to another.

\section{Inter-Kernel Messaging Layer}
Popcorn Linux comes with a high efficient messaging layer for inter-kernel communication ~\cite{shelton2013popcorn}.
% Important fact: messaging layer is strictly FIFO

\section{Popcorn Namespace}
\subsection{Replicated Execution}
\subsection{FT PID}
\section{Network Stack Replication}
% Important fact: accept sequence is same on primary and replica